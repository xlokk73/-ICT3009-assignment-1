\documentclass[a4paper, 12pt]{article}
\usepackage{graphicx}
\usepackage{listings, xcolor}
\usepackage{fancyvrb}


\definecolor{lightgray}{rgb}{.9,.9,.9}
\definecolor{darkgray}{rgb}{.4,.4,.4}
\definecolor{purple}{rgb}{0.65, 0.12, 0.82}
\lstdefinelanguage{JavaScript}{
	keywords={typeof, new, true, false, catch, function, return, null, catch, switch, var, if, in, while, do, else, case, break},
	keywordstyle=\color{blue}\bfseries,
	ndkeywords={class, export, boolean, throw, implements, import, this},
	ndkeywordstyle=\color{darkgray}\bfseries,
	identifierstyle=\color{black},
	sensitive=false,
	comment=[l]{//},
	morecomment=[s]{/*}{*/},
	commentstyle=\color{purple}\ttfamily,
	stringstyle=\color{red}\ttfamily,
	morestring=[b]',
	morestring=[b]"
}

\definecolor{verylightgray}{rgb}{.97,.97,.97}
\lstdefinelanguage{Solidity}{
	keywords=[1]{anonymous, assembly, assert, balance, break, call, callcode, case, catch, class, constant, continue, constructor, contract, debugger, default, delegatecall, delete, do, else, emit, event, experimental, export, external, false, finally, for, function, gas, if, implements, import, in, indexed, instanceof, interface, internal, is, length, library, log0, log1, log2, log3, log4, memory, modifier, new, payable, pragma, private, protected, public, pure, push, require, return, returns, revert, selfdestruct, send, solidity, storage, struct, suicide, super, switch, then, this, throw, transfer, true, try, typeof, using, value, view, while, with, addmod, ecrecover, keccak256, mulmod, ripemd160, sha256, sha3}, % generic keywords including crypto operations
	keywordstyle=[1]\color{blue}\bfseries,
	keywords=[2]{address, bool, byte, bytes, bytes1, bytes2, bytes3, bytes4, bytes5, bytes6, bytes7, bytes8, bytes9, bytes10, bytes11, bytes12, bytes13, bytes14, bytes15, bytes16, bytes17, bytes18, bytes19, bytes20, bytes21, bytes22, bytes23, bytes24, bytes25, bytes26, bytes27, bytes28, bytes29, bytes30, bytes31, bytes32, enum, int, int8, int16, int24, int32, int40, int48, int56, int64, int72, int80, int88, int96, int104, int112, int120, int128, int136, int144, int152, int160, int168, int176, int184, int192, int200, int208, int216, int224, int232, int240, int248, int256, mapping, string, uint, uint8, uint16, uint24, uint32, uint40, uint48, uint56, uint64, uint72, uint80, uint88, uint96, uint104, uint112, uint120, uint128, uint136, uint144, uint152, uint160, uint168, uint176, uint184, uint192, uint200, uint208, uint216, uint224, uint232, uint240, uint248, uint256, var, void, ether, finney, szabo, wei, days, hours, minutes, seconds, weeks, years},	% types; money and time units
	keywordstyle=[2]\color{teal}\bfseries,
	keywords=[3]{block, blockhash, coinbase, difficulty, gaslimit, number, timestamp, msg, data, gas, sender, sig, value, now, tx, gasprice, origin},	% environment variables
	keywordstyle=[3]\color{violet}\bfseries,
	identifierstyle=\color{black},
	sensitive=false,
	comment=[l]{//},
	morecomment=[s]{/*}{*/},
	commentstyle=\color{gray}\ttfamily,
	stringstyle=\color{red}\ttfamily,
	morestring=[b]',
	morestring=[b]"
}

\setlength\parindent{24pt}

\lstset{language=python,breaklines=true, frame=single}

\begin{document}
\begin{figure}
    \centering
    \includegraphics[width=1\textwidth]{Logo}
\end{figure}

\title{Assignment 1 Report}
\author{Manwel Bugeja}
\date{\today}
\maketitle
  
\tableofcontents
\newpage

\section{Problem 1}

\section{Flow of the Contract}

\begin{figure}[h!]
	\centering
	\includegraphics[width=\textwidth]{./Diagrams/problem1-flowchart}
	\caption{Submitting a Loan Request}
	\label{fig:flowchart1}
	% reference with \ref{fig:mesh1}
\end{figure}


\begin{figure}[h!]
	\centering
	\includegraphics[width=\textwidth]{./Diagrams/problem1-flowchart2}
	\caption{Submitting a Loan}
	\label{fig:flowchart2}
	% reference with \ref{fig:mesh1}
\end{figure}

The flow of the process is shown via a flowchart in figures \ref{fig:flowchart1} and \ref{fig:flowchart2}.


\subsection{Structures}

The most fundamental part of the contract is a structure for loan requests, \textit{LoanRequest}. 
This can be seen in listing \ref{lst:LoanRequestStruct}. This structure contains the information required regarding the three persons included in the deal, i.e. the borrower, the guarantor and the loaner.

\begin{lstlisting}[caption={Loan Request Structure}\label{lst:LoanRequestStruct}, basicstyle=\ttfamily, frame=single, language=Solidity]
struct LoanRequest{
	uint    creationDate;       
	
	address payable borrower;
	uint    amount;             // Amount to be borrowed
	uint    expiryTime;         // Max time (in seconds) the loan will be paid back
	uint    interestPaid;       // Amount (in wei) paid on loan payback
	
	address payable guarantor; 
	uint    guarantorInterest;  // Amount (in wei) taken from interest by the guarantor
	
	address payable loaner;     
	State   state;            
}
\end{lstlisting}

The last field of the loan request structure is an enumeration called \textit{State}, which can be seen in listing \ref{lst:StateEnum}. This enumeration contains the information about what state the loan request resides in. The states are explained in table \ref{tab:StateEnum}.


	
\begin{lstlisting}[caption={Loan Request Structure}\label{lst:StateEnum}, basicstyle=\ttfamily, frame=single, language=Solidity]
enum State {
	REQUESTED,
	PENDING,
	GUARANTEED,
	LOANED,
	PAID,
	TERMINATED
}
\end{lstlisting}

\begin{center}
	\resizebox{\textwidth}{!}{\begin{tabular}{ |c| c| }
			\hline
			\textbf{State} & \textbf{Description} \\ 
			\hline
			REQUESTED    & The borrower submitted a request  \\  
			\hline
			PENDING         &The guarantor submitted a guarantee and is waiting for approval \\
			\hline
			GUARANTEED & The guarantee has been accepted by the borrower\\
			\hline
			LOANED          & The loaner has loaned the money\\
			\hline
			PAID                & The borrower paid the loaner back before the loan expired\\
			\hline
			TERMINATED   & The loan expired and the loaner took the guarantee\\
			\hline
	\end{tabular}}
	\label{tab:StateEnum}
\end{center}

These loan requests are all kept in a dynamic array, \textit{LoanRequests[]}. Elements are added to this array from the functions. However, elements are never removed. This allows the system to keep a log of all requests ever made.

\subsection{Functions}

Functions in the contract are split into four sections, these being: general purpose, borrower, guarantor and loaner functions.
\\
General purpose functions are mostly used to get information about the loan requests. This means that most of them include the \textit{view} keyword.
\\
Functions that act on a created loan request take the index as a parameter. Furthermore, the functions have the following flow: checks, followed by code that transfer balance, followed by updating the loan request states. The reason for this is explain in the section about security. 



\section{Problem 2}
\section{Problem 3}
\section{Problem 4}
\section{Problem 5}

%\bibliographystyle{abbrv}
 %\bibliography{references}

\end{document}
